\chapter{Kapitel 2: Spielregeln}
\section{Würfel}
In DnD werden die Würfel 1 x W4, 1 x W6, 1 x W8, 2 x W10, 1 x W12 und 1 x W20 verwendet. Hierbei steht das W für Würfel (im Englischen meist D) und die Zahl für die Seitenzahl des Würfels.
\section{Spielablauf}
Eine Gruppe Abenteurern beschreiten ein Abenteuer, dass der Dungeon Master beschreibt. Der Ablauf einer Runde ist wie folgt: \\ \noindent \textbf{1) Der Spielleiter beschreibt die Umgebung} und erklärt die Grundlegenden Handlungsmöglichkeiten. \\
\noindent \textbf{2) Die Spieler beschreiben, was sie tun wollen}
\noindent Manchmal sind die Handlungen leicht und der DM wird die Konsequenzen der Handlung oder den Ort beschreiben, aber bei schwierigen Handlungen wird der Dungeon Master einen Würfelwurf verlangen und selbst den Ausgang der Handlung bestimmen. \\
\noindent \textbf{3) Der Spielleiter beschreibt die Auswirkungen der Spielerhandlungen.} Dies bringt die Spieler wieder zu Punkt 1).\\
\noindent  Im Kampf ist dieser Ablauf strukturierter, meisten jedoch passt sich der Ablauf flüssig an die Situationen an. Einige Dungeon Master verwenden auch Musik und/oder Miniaturfiguren.

\section{Attribute}
Spieler-Charaktere (PC) oder Nicht-Spieler-Charaktere (NPC) darunter auch Monster und zB Dorfbewohner haben alle \textbf{sechs verschiedene Attribute}:

\begin{itemize}
  \item Stärke
  \item Geschicklichkeit
  \item Konstitution
  \item Intelligenz
  \item Weisheit
  \item Charisma
\end{itemize}


\begin{minipage}{.45\linewidth}
	\begin{dndtable}
	   	\textbf{Punktzahl}  & \textbf{Modifikator} \\
			1 & -5 \\
			2-3 & -4 \\
			4-5 & -3 \\
			6-7 & -2 \\
			8-9 & -1 \\
			10-11 & +0 \\
			12-13 & +1 \\
			14-15 & +2 \\
	\end{dndtable}
\end{minipage}
\begin{minipage}{.45\linewidth}
	\begin{dndtable}
	   	\textbf{Punktzahl}  & \textbf{Modifikator} \\
			16-17 & +3 \\
			18-19 & +4 \\
			20-21 & +5 \\
			22-23 & +6 \\
			24-25 & +7 \\
			26-27 & +8 \\
		  28-29 & +9 \\
			30 & +0 \\
	\end{dndtable}
\end{minipage}

\section{Die Grundregeln}
\noindent Um das Resultat einer Handlung bestimmt werden soll, wird in DnD ein W20 verwendet. Dies gilt also für Attributs-, Angriffs-, und Rettungswürfe. \\
\noindent Der Ablauf:\\

\subsubsection*{}

\noindent \textbf{1) Würfeln}\\
Rechne den Modifikator drauf \\
\noindent \textbf{2) Bonus und Malus bedenken}\\
Ein Zauber, besondere Umstände- oder \\ andere Effekte\\
\noindent \textbf{3) Vergleichen}\\
Wenn der Wert den Minimalwert erreicht oder \\ übersteigt, dann ist der Wurf ein Erfolg.\\

\subsubsection*{}

\noindent Der vorgegebene Wert eines Attributs oder Rettungswurf ist der \textit{Schwierigkeitsgrad (SG)}. Für Angriffe ist es die \textit{Rüstungsklasse (RK)}. Dies wird alles später nochmal erwähnt.

\subsection{Vorteil und Nachteil}
Manchmal hast du einen Vorteil oder einen Nachteil auf einen Zauber oder eine Aktion. Dafür wirfst du zwei Würfel. Je nachdem ob Vorteil oder Nachteil verlangt wird, verwendest du den besten oder den schlechtesten Wurf der Beiden Würfe.
Falls du sowohl Nachteil als auch Vorteil auf eine Aktion hast, so verwendest du weder Vorteil noch Nachteil, auch wenn du zB zwei mal Vorteil und einmal Nachteil auf die Aktion hast. Mit dem Merkmal \textit{Halblingsglück} kann ein Held einen der Beiden noch einmal werfen.

\subsection{Attributswürfe}
Ein Attributswurf prüft das natürliche Talent und das Training eines Charakters oder Monsters, um eine Herausforderung zu bestehen.
Der DM kann Attributswürfe verlangen wenn eine Handung gelingen oder Misslingen kann (außer Anriffe).
Der Held würfelt dann mit einem W20 und wenn der Wurf nach dem einberechnen des Bonus / Malus größer / gleich dem SG ist, war der Wurf ein Erfolg.

\subsubsection{Übungsbonus}
Du kannst in einer Aufgabe, die mit einem Attributswurf zusammenhängt besonders gut sein. Du kannst nie mehr als eine Kompetenz einmal auf einen Wurf verrechnen.

\subsubsection{Vergleiche}
Wenn zwei Gestalten zB. ein Goblin (NPC) und ein Halbling (PC) gleichzeitig eine Aktion ausführen und sie nur einem von beiden gelingen kann, dass der Spielleiter eine Vergleichsprobe verlangt. Beide Charaktere werfen einen Wert und vergleichen ihn. Bei Unentschieden haben beide verloren. (Beispiel: Trinkspiele)

\subsubsection{Fertigkeiten}
Jedes Attrubut deckt eine breite Reihe von Fähigkeiten ab, darunter auch Fertigkeiten. in welcher ein Charakter geübt sein kann. Eine Fertigkeit repräsentiert einen bestimmten Aspekt eines Attributswertes: die Übung eines Charakters in einer Fertigkeit zeigt einen Scherpunkt auf diesen Aspekt.
Beispiel: Ein Charakter der die Fertigkeit Heimlichkeit besitzt hat Vorteile bei Würfen die mit Schleichen und Vertstecken zu tun haben. Die Folgenden Fertigkeiten gehören zu den jeweiligen Attributen.\\

\noindent \textbf{Stärkewurf}\\
\begin{itemize}
  \item Athletik
\end{itemize}
\noindent \textbf{Geschicklichkeitswurf}\\
\begin{itemize}
  \item Akrobatik
  \item Fingerfertigkeit
  \item Heimlichkeit
\end{itemize}
\noindent \textbf{Konstitutionswurf}\\
\begin{itemize}
  \item -
\end{itemize}
\noindent \textbf{Intelligenzwurf}\\
\begin{itemize}
  \item Arkane Kunde
  \item Geschichte
  \item Nachforschungen
  \item Naturkunde
  \item Religion
\end{itemize}
\noindent \textbf{Weisheitswurf}\\
\begin{itemize}
  \item Mit Tieren umgehen
  \item Motiv erkennen
  \item Heilkunde
  \item Wahrnehmung
  \item Überlebenskunst
\end{itemize}
\noindent \textbf{Charisma}\\
\begin{itemize}
  \item Täuschen
  \item Einschüchtern
  \item Auftreten
  \item Überzeugen
\end{itemize}

\subsection{Rettungswürfe}
EIn Rettungswurf stellt einen Versuch dar einem Zauber / einer Falle / einem Gift / einer Krankheit oä. zu widerstehen. Dazu würfelst du mit einem W20. Ein Rettungswurf kann auch durch Situationsbedingte Boni und Mali modifiziert werden.
